\RequirePackage[l2tabu, orthodox]{nag}

\documentclass[12pt, a4paper]{article}

\usepackage[english]{babel}       % Voor nederlandstalige hyphenatie (woordsplitsing)

\usepackage{amsmath}                    % Uitgebreide wiskundige mogelijkheden
\usepackage{amssymb}                    % Voor speciale symbolen zoals de verzameling Z, R...

\usepackage{microtype}

\usepackage[super]{nth}

\usepackage{booktabs}

\usepackage{url}                        % Om url's te verwerken
\usepackage{graphicx}                   % Om figuren te kunnen verwerken

\usepackage[utf8]{inputenc}           
\usepackage[T1]{fontenc}  

\usepackage{float}                      % Om nieuwe float environments aan te maken. Ook optie H!
\usepackage{listings}                   % Voor het weergeven van letterlijke text en codelistings
\usepackage{inconsolata}

\usepackage{caption}
\usepackage{subcaption}

\usepackage{multirow}

\lstset{
  belowcaptionskip=1\baselineskip,
  breaklines=true,
  frame=L,
  xleftmargin=\parindent,
  language=Lisp,
  showstringspaces=false,
  basicstyle=\footnotesize\ttfamily,
  keywordstyle=\bfseries,
  commentstyle=\itshape,
}

\usepackage{geometry}
\geometry{
  top=1in,     
  inner=1in,
  outer=1in,
  bottom=1in,
  headheight=3ex,
  headsep=2ex,
}

\usepackage{fancyhdr}                   % Voor fancy headers en footers.
\fancyhf{}
\fancyhead[L]{\slshape Bayesian Networks}
\fancyhead[R]{\slshape Tom Naessens}
\fancyfoot[C]{\thepage}

   
\setlength{\headheight}{15pt} % fixes \headheight warning

\usepackage[dutch]{varioref}
\usepackage[plainpages=false, hidelinks]{hyperref}    % Om hyperlinks te hebben in het pdfdocument. 
 
\graphicspath{{includes/}}               % De plaats waar latex zijn figuren gaat halen.

\renewcommand{\thesubsection}{Q.\arabic{subsection}}

\begin{document}
\selectlanguage{english}

\begin{titlepage}

\fontsize{12pt}{14pt}
\selectfont

\begin{center}
\includegraphics[height=3cm]{UGent}

\vspace{1.5cm}
Ghent University \\
Faculty of Engineering and Architecture \\
Knowledge Based Systems and Artificial Intelligence \\


\vspace{4.0cm}

\fontseries{bx}
\fontsize{17.28pt}{21pt}
\selectfont

\textsc{{\Large Neural Networks}}

\fontseries{m}
\fontsize{12pt}{14pt}
\selectfont

\vspace{.6cm}

{\Large 
	Tom \textsc{Naessens} \hfill \\
} 

\vspace{.6cm}

{ \large
	\nth{2} master Computer Science Engineering
}

\vspace{0.4cm}

Friday December 19, 2014
\end{center}

\vspace{5.5cm}


\end{titlepage}

\thispagestyle{empty}


\pagestyle{fancy}

\section{Bayesian Network}
The Bayesian network can be found in the attached file (\texttt{BayesianNetworks\_Naessens\_Tom.xbnc}).

An assumption made is that, when a person has two diseases and develops a third which is a result of those two diseases, only one of those diseases was the cause of the third disease. This leads to the following equation: $\Pr(X \mid Y \cap Z) = \Pr(X \mid Y) + (1 - \Pr(X \mid Y) \cdot \Pr(X \mid Z)) = \Pr(X \mid Z) + (1 - \Pr(X \mid Z) \cdot \Pr(X \mid Y))$. In \ref{q2}, this reasoning is worked out more thoroughly.

\section{Answers to questions}
\subsection*{Disclaimer}
During the lab session, I have worked together with Dieter Decaestecker, so it is possible that our results overlap or are similar.

To keep some proofs short, some variable names are shortened to their initials. ``Weaked Immune System'' can for example be shortened to ``wis'', ``Lung Cancer'' to ``lc'' etc.

\subsection{Which nodes are the observation (evidence) nodes, and which are the query nodes?}
We make the following distinction:
\begin{itemize}
\item \textbf{Hypothesis nodes:} Bronchitis, Lung Cancer, Pneumonia, Weakened Immune System, Common Cold
\item \textbf{Informational nodes:} Smoking, Runny Nose, Fever, Chest Pain, Shortness Of Breath, Loose Cough
\end{itemize}

\subsection{Provide the conditional probability table corresponding to the node `pneumonia'.}
\label{q2}.
The risk factors dictate that 30\% of the people with a weakened immune system, and 5\% of the people with lung cancer are diagnosed with pneumonia, so $\Pr(Pneumonia \mid  WeakedImmuneSystem) = 0.30$ and $\Pr(Pneumonia \mid LungCancer) = 0.05$. This only accounts for people with pneumonia having one of both diseases, and not both. To calculate the chance to have pneumonia when a patient has lung cancer and a weakened immune system, we assume that if a person with both diseases develops pneumonia, only one of these diseases was the cause of pneumonia. This assumption leads to the following derivation:

\begin{align*}
\Pr(pn \mid wis \cap lc) &= \Pr(pn \mid lc) + \left(1 - \Pr\left(pn \mid lc\right)\right) \cdot \Pr(pn \mid wis)\\
&= 0.05 + (1 - 0.05) * 0.3\\
&= 0.335
\end{align*}

Which results in $\Pr(Pneumonia \mid wis \cap lc) = 0.335$.

Last but not least, the chance to have pneumonia without having a weakened immune system or lung cancer is given in the text: 0.1\%, resulting in: \\ $\Pr\left(Pneumonia \mid \overline{WeakedImmuneSystem \cup LungCancer}\right) = 0.001$.

We can now fill in our conditional probability table for `pneumonia', the result of which can be found in table \ref{tab:pneumonia}.

\begin{table}[ht]
\centering
\caption{Conditional probability table for `pneumonia'}
\label{tab:pneumonia}
\begin{tabular}{ccll}
\toprule
\multicolumn{2}{c}{Parent Nodes}                                             & \multicolumn{2}{c}{Pneumonia} \\ \cmidrule(r){1-2} \cmidrule(l){3-4}
\multicolumn{1}{l}{Weakened Immune System} & \multicolumn{1}{l}{Lung Cancer} & Yes           & No            \\ \midrule
\multirow{2}{*}{Yes}                       & Yes                             & 0.335          & 0.25          \\ 
                                           & No                              & 0.30          & 0.70          \\ 
\multirow{2}{*}{No}                        & Yes                             & 0.05          & 0.95          \\ 
                                           & No                              & 0.001         & 0.999         \\
\bottomrule
\end{tabular}
\end{table}

\subsection{In the case of high fever without a runny nose, pneumonia must be considered. Why?}
Because pneumonia has a high chance ($\Pr(Pneumonia | Fever \cap \overline{RunnyNose}) =$) of $0.5190$ to be the cause of the high fever.

\subsection{Lung cancer is often found in patients with chest pain, shortness of breath, no fever, and usually no loose cough.}
This is correct. The chance that lung cancer is the cause of these symptons is $0.8786$ ($= \Pr(LungCancer | ChestPain \cap ShortnessOfBreath \cap \overline{Fever} \cap \overline{LooseCough})$) .

\subsection{A patient suffers high fever and a severe loose cough (although he is not a smoker). He speaks nasally. What is your diagnosis (a weakened immune system may be excluded)?}
Assuming that `he speaks nasally' means that the patient has a runny nose, we obtain the following probabilities:

\begin{align*}
\Pr(CommonCold | Fever \cap LooseCough \cap RunnyNose \cap \overline{Smoking}) &= 0.9920 \\
\Pr(Bronchitis | Fever \cap LooseCough \cap RunnyNose \cap \overline{Smoking}) &= 0.0544 \\
\Pr(Pneumonia | Fever \cap LooseCough \cap RunnyNose \cap \overline{Smoking}) &= 0.3947 \\
\Pr(LungCancer | Fever \cap LooseCough \cap RunnyNose \cap \overline{Smoking}) &= 0.0212 \\
\end{align*}

The chance for a common cold is almost 100\%, so we can safely conclude that the patient just has a common cold.

\subsection{The center for homeless people called. An unknown person who does not look very good has arrived and seeks help. He has high fever and loose cough. What is your primary diagnosis? What additional information would be necessary to provide a more confident diagnosis?}
The initial chances we obtain are the following:
\begin{align*}
\Pr(CommonCold | Fever \cap LooseCough) &= \underline{0.6220} \\
\Pr(Pneumonia | Fever \cap LooseCough) &= 0.5233 \\
\Pr(WeakenedImmuneSystem | Fever \cap LooseCough) &= 0.4625 \\
\Pr(Bronchitis | Fever \cap LooseCough) &= 0.2869 \\
\Pr(LungCancer | Fever \cap LooseCough) &= 0.0962 \\
\end{align*}
Our first and quick hypothesis would be that the homeless person has a common cold, but as the chances for pneumonia or a weakened immune system are more than likely too, it would be better to try to find more symptoms.

If the homeless person also has a runny nose, it becomes clear that he just has a common cold:
\begin{align*}
\Pr(CommonCold | Fever \cap LooseCough \cap RunnyNose) &= \underline{0.9936} \\
\Pr(Pneumonia | Fever \cap LooseCough \cap RunnyNose) &= 0.3025 \\
\Pr(WeakenedImmuneSystem | Fever \cap LooseCough \cap RunnyNose) &= 0.2823 \\
\Pr(Bronchitis | Fever \cap LooseCough \cap RunnyNose) &= 0.3694 \\
\Pr(LungCancer | Fever \cap LooseCough \cap RunnyNose) &= 0.0800 \\
\end{align*}

Without a runny nose, the probabilities lie more in the favour of pneumonia, in combination with or possibly caused by a weakened immune system:
\begin{align*}
\Pr(CommonCold | Fever \cap LooseCough \cap \overline{RunnyNose}) &= 0.0767 \\
\Pr(Pneumonia | Fever \cap LooseCough \cap \overline{RunnyNose}) &= \underline{0.8473} \\
\Pr(WeakenedImmuneSystem | Fever \cap LooseCough \cap \overline{RunnyNose}) &= \underline{0.7268} \\
\Pr(Bronchitis | Fever \cap LooseCough \cap \overline{RunnyNose}) &= 0.1658 \\
\Pr(LungCancer | Fever \cap LooseCough \cap \overline{RunnyNose}) &= 0.1199 \\
\end{align*}

If we add chest pain to the initial mix of symptoms, the chances remain the same as previous probabilities (in the favour of pneumonia and a weakened immune system):
\begin{align*}
\Pr(CommonCold | Fever \cap LooseCough \cap ChestPain) &= 0.4272 \\
\Pr(Pneumonia | Fever \cap LooseCough \cap ChestPain) &= \underline{0.8711} \\
\Pr(WeakenedImmuneSystem | Fever \cap LooseCough \cap ChestPain) &= \underline{0.7461} \\
\Pr(Bronchitis | Fever \cap LooseCough \cap ChestPain) &= 0.1676 \\
\Pr(LungCancer | Fever \cap LooseCough \cap ChestPain) &= 0.1601 \\
\end{align*}

If we however add ShortnessOfBreath to the equation, we don't get much results as all the chances are in the neighbourhood of 0.5.
\begin{align*}
\Pr(CommonCold | Fever \cap LooseCough \cap ShortnessOfBreath) &= 0.5807 \\
\Pr(Pneumonia | Fever \cap LooseCough \cap ShortnessOfBreath) &= 0.5971 \\
\Pr(WeakenedImmuneSystem | Fever \cap LooseCough \cap ShortnessOfBreath) &= 0.5174 \\
\Pr(Bronchitis | Fever \cap LooseCough \cap ShortnessOfBreath) &= 0.4980 \\
\Pr(LungCancer | Fever \cap LooseCough \cap ShortnessOfBreath) &= 0.1321 \\
\end{align*}

Combining all the previous tests, it would be best to check for a runny nose as this eliminates all other diseases and is easy to check. If a runny nose can't be established, it should be best to check for chest pain. In any chance, lung cancer can be excluded.


\subsection{A colleague calls. One of his patients suffers from a recurrent pneumonia. This patient is a heavy smoker, but for the rest, he has a normal life. What is your advice?}
If we take all inputs in consideration, we become the following chances:
\begin{align*}
\Pr(WeakenedImmuneSystem | Smoking \cap Pneumonia \cap \overline{ChestPain} \cap \overline{Fever} \\ \cap \overline{LooseCough} \cap \overline{RunnyNose} \cap \overline{ShortnessOfBreath}) &=& 0.7303 \\
\Pr(LungCancer | Smoking \cap Pneumonia \cap \overline{ChestPain} \cap \overline{Fever} \\ \cap \overline{LooseCough} \cap \overline{RunnyNose} \cap \overline{ShortnessOfBreath}) &=& 0.3092
\end{align*}

Judging by the facts, the cause of the patients complaints are probably the cause of a weakened immune system. The chance for lung cancer given this information is not very low either. Given the seriousness of lung cancer, the patient should get checked for this too. In any case, the patient should stop smoking for his own health, and the health of the people around him.

\end{document}
