\documentclass[11pt,a4paper]{article}
\usepackage[latin1]{inputenc}
\usepackage[dutch]{babel}
\title{Installation Manual Bebras (nl)}
\author{Team Biber}

\usepackage{fullpage}
\usepackage{hyperref}
\usepackage{url}
\usepackage[pdftex]{graphicx}

\setcounter{secnumdepth}{5}
\setcounter{tocdepth}{5}
\makeindex
\begin{document}
\maketitle
\parindent 0pt

\section{Dependencies}
\begin{itemize}
\item Make sure you have a working \href{http://www.playframework.com/documentation/2.1.0/Home}{Play installation}. We use Play 2.1.0!
\item Make sure you have a working \href{http://www.oracle.com/technetwork/java/javase/downloads/index.html}{Java installation}. We use Java 7!
\item Get our codebase \href{https://github.ugent.be/tnnaesse/Biber}{here}.
\item If you don't have a working Git installation, remove the file \texttt{project/project/plugin.scala}
\item Firefox needs to be installed in order for the Seleniumtests to work! ATTENTION: It is possible that a Seleniumtests fails due to delay. We set a specific waiting delay ranging from 300ms to 3 seconds according to the page load. If a test fail, please repeat it using \texttt{play 'test-only [name test]'}.
\end{itemize}

\section{Testing}
\begin{itemize}
\item Open a terminal
\item Navigate to the root of the codebase
\item Navigate to the \texttt{Biber/} folder
\item Execute \texttt{play test}
\end{itemize}

\section{Developer mode}
\begin{itemize}
\item Open a terminal
\item Navigate to the root of the codebase
\item Navigate to the \texttt{Biber/} folder
\item Execute \texttt{play run}; the developer mode works with an H2 in memory database so it is not needed to configure one
\item Navigate in your browser to \texttt{localhost:9000}
\end{itemize}

\section{Production mode}
Before you begin; make sure you have a working Postgres installation with a predefined user

\begin{itemize}
\item Open a terminal
\item Navigate to the root of the codebase
\item Navigate to the \texttt{Biber/} folder
\item Execute \texttt{play test}
\item Log into your Postgres installation with the prefered user
\item Import all the files in \texttt{conf/evolutions/prod} in order (first \texttt{1.sql} then \texttt{2.sql} and so on). After these files have been added, import \texttt{admin.sql} to generate the admin user
\item Edit \texttt{conf/prod.conf} and edit the details of your Postgres installation to your needs
\item Execute \texttt{play -Dconfig.file=conf/prod.conf start}
\item Navigate to \texttt{localhost:9000/play/}
\end{itemize}


\end{document}