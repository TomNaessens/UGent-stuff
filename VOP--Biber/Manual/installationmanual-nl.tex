\documentclass[11pt,a4paper]{article}
\usepackage[latin1]{inputenc}
\usepackage[dutch]{babel}
\title{Installation Manual Bebras (en)}
\author{Team Biber}

\usepackage{fullpage}
\usepackage{hyperref}
\usepackage{url}
\usepackage[pdftex]{graphicx}

\setcounter{secnumdepth}{5}
\setcounter{tocdepth}{5}
\makeindex
\begin{document}
\maketitle
\parindent 0pt

\section{Dependencies}
\begin{itemize}
\item Zorg ervoor dat je een werkende \href{http://www.playframework.com/documentation/2.1.0/Home}{Play installatie}. Wij gebruiken Play 2.1.0!
\item Zorg ervoor dat je een werkende \href{http://www.oracle.com/technetwork/java/javase/downloads/index.html}{Java installatie} hebt. Wij gebruiken Java 7!
\item Haal de code vanop deze repository \href{https://github.ugent.be/tnnaesse/Biber}{hier}.
\item Als je geen werkende Git installatie hebt, verwijder dan \texttt{project/project/plugin.scala}
\item Om de seleniumtests te doen slagen moet Firefox geïnstalleerd zijn! OPGELET: Het KAN zijn dat bij het testen enkele tests niet slagen door te veel delay. Wij stellen overal een maximum delay van 300ms tot 3 seconden in, afhankelijk van de pagina. Dit is in het merendeel van de gevallen genoeg gebleken. Als er een Seleniumtest faalt kan deze herdaan worden met \texttt{play 'test-only [naam test]'}.
\end{itemize}

\section{Testing}
\begin{itemize}
\item Open een terminal
\item Navigeer naar de hoofdmap van onze code
\item Navigeer naar de \texttt{Biber/} map
\item Voer \texttt{play test} uit
\end{itemize}

\section{Developer mode}
\begin{itemize}
\item Open een terminal
\item Navigeer naar de hoofdmap van onze code
\item Navigeer naar de \texttt{Biber/} map
\item Voer \texttt{play test} uit
\item Voer \texttt{play run} uit; de developer modus werkt met een 'in memory' databank
\item Navigeer naar \texttt{localhost:9000}
\end{itemize}

\section{Production mode}
Voor je begint: zorg er eerst voor dat je een werkende Postgresinstallatie hebt met een voorgeconfigureerde gebruiker!

\begin{itemize}
\item Open een terminal
\item Navigeer naar de hoofdmap van onze code
\item Navigeer naar de Biber map
\item Log in op uw Postgres installation met uw gebruiker naar keuze
\item Importeer alle bestanden uit de map \texttt{conf/evolutions/prod} in volgorde (eerst \texttt{1.sql}, dan \texttt{2.sql} enzovoort). Laad hierna het script \texttt{admin.sql} in om de admingebruiker in de databank te zetten
\item Pas \texttt{conf/prod.conf} aan en wijzig de databasedetails naar uw Postgresconfiguratie en -gebruiker
\item Voer \texttt{play -Dconfig.file=conf/prod.conf start}
\item Navigeer naar \texttt{localhost:9000/play/}
\end{itemize}


\end{document}