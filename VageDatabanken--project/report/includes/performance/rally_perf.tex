\subsection{Drift}
Omdat de driftcontroller gebaseerd is op de snelle controller zien we 
gelijkaardige resultaten. Door te driften zien we echter voor Interlagos een 
betere tijd opduiken.

Zoals besproken bij de implementatie van deze controller is deze controller een 
gewaagde versie van de speedcontroller waarbij we meer risico's nemen zoals 
later remmen, oversturen, meer gas geven bij het verlaten van de bocht, etc. 
Daarom zien we ook dat deze controller meer crasht dan de speedcontroller. 

Het parcour op Interlagos is perfect voor deze controller. Door de wijde 
bochten waarvan de hoek de hele bocht gelijk blijft drift hij zich hier vaak 
mooi door. Omdat we sneller door de bochten durven gaan de speedcontroller, 
alsook op de rechte stukken goed gas durven geven halen we hier zelfs een 
snellere tijd dan bij de speedcontroller, al ligt de topsnelheid een stuk 
lager. Bij de lange stukken durft hij wel eens de controle over het stuur 
te verliezen door kleine stuurcorrecties, of niet genoeg afremmen bij het 
insturen van een bocht. Ditzelfde gedrag zien we ook vaak bij Texas de kop 
opsteken.

Als we naar de tabel kijken in Tabel~\ref{tbl:resultsdrift} zien we dat de 
wagen ook effectief meer drift dan de andere controllers.

Op het parcours van Silverstone en Spa-Francorchamps presteert deze controller 
veel minder goed en mogen we al blij zijn dat hij (in het geval van 
Silverstone) het parcours uitrijdt. De wagen probeert zich vaak door bochten te 
driften, maar aangezien die vaak te scherp zijn heeft hij hier meestal de 
snelheid niet voor.

De robuustheid van deze controller ligt nog lager dan die van die van de 
speedcontroller. Rechte stukken worden soms snel genomen, soms niet. De ene 
keer drift hij goed door bochten, waar hij de volgende keer traag zonder 
driften door rijdt. De oorzaak hiervan is dat het gedrag een stuk minder 
gecontroleerd en een stuk meer bruusk is dan bij andere controllers.