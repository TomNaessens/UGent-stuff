\subsection{Snelheid}
De snelle controller is in staat om de circuits van Silverstone en Interlagos af
te leggen zonder te crashen. De snelheid zal hier vari\"eren van 80
km/h in de bochten tot 130 km/h op de rechte stukken. Bij lange rechte stukken
is het mogelijk dat de wagen hogere snelheden behaalt, de topsnelheid die wij op
deze circuits waargenomen hebben is $304.16$ km/h.

Op het snelle circuit van Texas zal de controller zeer hoge snelheden behalen
maar niet op tijd beginnen remmen. De controller zal dus tegen een hoge snelheid
de bocht proberen nemen en bijgevolg slippen en crashen of een lus maken. Ondanks de crash zal
deze controller zeer snelle tijd neerzetten op het circuit. De topsnelheid die we hier meten is $352.73$ km/h.

Zoals we in Tabel~\ref{tbl:resultsspeed} kunnen zien zal de snelle controller ook aanleg hebben tot driften maar zal hier niet op doelen. Zo zien we dat deze op het circuit van Interlagos zijn snelste ronde neerzet zonder te driften. Ook op Silverstone blijft de drift beperkt. Door de crash in Texas zal hier natuurlijk een hoge driftscore verschijnen.

De robuustheid van deze controller ligt al een stuk lager dan die van de veilige controller. Wanneer de wagen namelijk uit een bocht komt en een lang recht stuk nadert, zal de wagen enkel sterk versnellen indien de wagen op tijd stabiel genoeg is. Indien de wagen maar net stabiel genoeg is zal de wagen mogelijks toch versnellen en dus heen en weer schommelen op het rechte stuk met een crash tot gevolg. De snelle controller zal dus ongeveer één op de tien keer crashen op de circuits van Silverstone en Interlagos.